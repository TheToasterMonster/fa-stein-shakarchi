\documentclass[letterpaper, 12pt]{article}
% \usepackage[margin=0.5in]{geometry}
\usepackage{fullpage}
\usepackage{microtype}
% \usepackage{kpfonts}
\usepackage{amsmath, amssymb, amsthm}
\usepackage{hyperref}
\usepackage{parskip}
\usepackage[many]{tcolorbox}
\usepackage{cancel}
\usepackage[shortlabels]{enumitem}

\setlist[enumerate]{(a)}
\tcbset{sharp corners, breakable, enhanced, parbox=false}

\setlength{\hoffset}{-1.54cm}
\setlength{\voffset}{-1.54cm}
\setlength{\topmargin}{0pt}
\setlength{\headsep}{0pt}
\setlength{\headheight}{0pt}
\setlength{\oddsidemargin}{0pt}
\setlength{\textwidth}{195mm}
\setlength{\textheight}{250mm}

\newcommand{\R}{\mathbb{R}}
\newcommand{\N}{\mathbb{N}}
\newcommand{\Z}{\mathbb{Z}}
\newcommand{\C}{\mathbb{C}}
\newcommand{\Q}{\mathbb{Q}}
\newcommand{\F}{\mathbb{F}}
\newcommand{\PP}{\mathbb{P}}
\newcommand{\Mod}[1]{\ {\mathrm{mod}\ #1}}
\newcommand{\Pmod}[1]{\ (\mathrm{mod}\ #1)}
\newcommand{\LHS}{\text{LHS}}
\newcommand{\RHS}{\text{RHS}}
\newcommand{\cm}{\checkmark}

\DeclareMathOperator{\tr}{tr}
\DeclareMathOperator{\nullspace}{null}

\title{Solutions to \textit{Fourier Analysis} by Stein and Shakarchi}
\author{Frank Qiang}
\date{Fall 2023}

\begin{document}
\maketitle
\tableofcontents

\section{Chapter 1: The Genesis of Fourier Analysis}
\subsection{Exercises}

\subsubsection{Exercise 1.1}
\begin{tcolorbox}
  If $z = x + iy$ is a complex number with $x, y \in \R$,
  we define
  \[|z| = (x^2 + y^2)^{1 / 2}\]
  and call this quantity the \textbf{modulus}
  or \textbf{absolute value} of $z$.
  \begin{enumerate}
    \item What is the geometric interpretation
      of $|z|$?
    \item Show that if $|z| = 0$, then $z = 0$.
    \item Show that if $\lambda \in \R$, then
      $|\lambda z| = |\lambda| |z|$, where $|\lambda|$
      denotes the standard absolute value of a real
      number.
    \item If $z_1$ and $z_2$ are two complex numbers,
      prove that
      \[
        |z_1 z_2| = |z_1| |z_2|
        \quad \text{and} \quad
        |z_1 + z_2| \le |z_1| + |z_2|.
      \]
    \item Show that if $z \ne 0$, then
      $|1 / z| = 1 / |z|$.
  \end{enumerate}
\end{tcolorbox}

\begin{proof}
  (a) The geometric interpretation of $|z|$ is the
  distance from $z$ as a vector in $\R^2$ to the origin.

  (b) If $|z| = 0$, then
  $|z|^2 = x^2 + y^2 = 0$.
  But $x^2, y^2 \ge 0$ since $x, y \in \R$, so
  $x^2 = y^2 = 0$. Thus $x = y = 0$.

  (c) If $\lambda \in \R$, then
  $\sqrt{\lambda^2} = |\lambda|$ and thus
  \[
    |\lambda z|
    = |\lambda x + i \lambda y|
    = \sqrt{(\lambda x)^2 + (\lambda y)^2}
    = \sqrt{\lambda^2 (x^2 + y^2)}
    = |\lambda| \sqrt{x^2 + y^2}
    = |\lambda| |z|.
  \]

  (d) Let $z_1 = x_1 + i y_1$ and $z_2 = x_2 + i y_2$.
  Then
  $z_1 z_2 = (x_1 x_2 - y_1 y_2) + i (x_1 y_2 + x_2 y_1)$
  and
  \[
    |z_1 z_2|
    = \sqrt{(x_1 x_2 - y_1 y_2)^2 + (x_1 y_2 + x_2 y_1)^2}
    = \sqrt{x_1^2 x_2^2 + y_1^2 y_2^2 + x_1^2 y_2^2 + x_2^2 y_1^2}
    = \sqrt{(x_1^2 + y_1^2)(x_2^2 + y_2^2)}
    = |z_1| |z_2|,
  \]
  where the mixed $x_1x_2y_1y_2$ terms cancel out.
  For the triangle inequality, we have
  \[
    |z_1 + z_2|^2
    = (x_1 + x_2)^2 + (y_1 + y_2)^2
    = x_1^2 + y_1^2 + x_2^2 + y_2^2 + 2(x_1 x_2 + y_1 y_2)
    = |z_1|^2 + |z_2|^2 + 2(x_1x_2 + y_1y_2)
  \]
  Now note that
  $x_1x_2 + y_1y_2 \le \sqrt{x_1^2 + y_1^2} \sqrt{x_2^2 + y_2^2} = |z_1| |z_2|$
  by the Cauchy-Schwarz inequality on $\R^2$, so
  \[
    |z_1 + z_2|^2
    \le |z_1|^2 + |z_2|^2 + 2|z_1| |z_2|
    = (|z_1| + |z_2|)^2.
  \]
  Taking square roots gives the desired result
  since the square root is monotonically increasing.

  (e) If $z \ne 0$, then $1 = (1 / z) z$ and thus
  \[
    1 = \left| \frac{1}{z} z \right|
    = \left| \frac{1}{z} \right| |z|
    \implies \left| \frac{1}{z} \right| = \frac{1}{|z|}
  \]
  by (c). This is valid since $|z| \ne 0$ by (b).
\end{proof}

\subsubsection{Exercise 1.2}
\begin{tcolorbox}
  If $z = x + iy$ is a complex number with $x, y \in \R$,
  we define the \textbf{complex conjugate} of $z$ by
  \[\overline{z} = x - iy.\]
  \begin{enumerate}
    \item What is the geometric interpretation
      of $\overline{z}$?
    \item Show that $|z|^2 = z \overline{z}$.
    \item Prove that if $z$ belongs to the unit circle,
      then $1 / z = \overline{z}$.
  \end{enumerate}
\end{tcolorbox}

\begin{proof}
  (a) The geometric interpretation of $\overline{z}$
  is a reflection of $z$ across the real axis.

  (b) We have
  $z \overline{z} = (x + iy)(x - iy) = x^2 - i^2 y^2
  = x^2 + y^2 = |z|^2$ since $i^2 = -1$.

  (c) If $z$ belongs to the unit circle, then
  $|z| = 1$. Then by (b),
  $1 = |z|^2 = z \overline{z}$, so
  $1 / z = \overline{z}$ since $z \ne 0$.
\end{proof}

\pagebreak
\section{Chapter 2: Basic Properties of Fourier Series}
\subsection{Exercises}
\subsubsection{Exercise 2.1}
\begin{tcolorbox}
  Suppose $f$ is $2\pi$-periodic and integrable on
  any finite interval. Prove that if $a, b \in \R$,
  then
  \[\int_a^b f(x)\, dx = \int_{a + 2\pi}^{b + 2\pi} f(x)\, dx = \int_{a - 2\pi}^{b - 2\pi} f(x)\, dx.\]
  Also prove that
  \[
    \int_{-\pi}^{\pi} f(x + a)\, dx
    = \int_{-\pi}^{\pi} f(x)\, dx
    = \int_{-\pi + a}^{\pi + a} f(x)\, dx.
  \]
\end{tcolorbox}

\begin{proof}
  For the first chain of equalities, make the
  change of variables $u = x + 2\pi$ with $du = dx$ to get
  \[
    \int_a^b f(x)\, dx = \int_{a + 2\pi}^{b + 2\pi} f(x - 2\pi)\, dx
    = \int_{a + 2\pi}^{b + 2\pi} f(x)\, dx
  \]
  since $f(x - 2\pi) = f(x)$ by periodicity (taking
  $y = x - 2\pi$ in $f(y + 2\pi) = f(y)$ yields this
  version).
  The second equality in the chain follows from
  using $u = x - 2\pi$. For the latter chain of
  equalities, first observe that the first and last
  integrals are equal by the change of variables
  $u = x + a$ with $du = dx$:
  \[
    \int_{-\pi}^{\pi} f(x + a)\, dx
    = \int_{-\pi + a}^{\pi + a} f(u)\, du.
  \]
  Now we show that the middle integral is equal to
  the last. Assume without loss of generality that
  $0 \le a \le 2\pi$ since $f$ is $2\pi$-periodic.
  Then we can write
  \[
    \int_{-\pi}^{\pi} f(x)\, dx
    = \int_{-\pi}^{-\pi + a} f(x)\, dx + \int_{-\pi + a}^{\pi} f(x)\, dx
  \]
  since $-\pi \le -\pi + a \le \pi$. Then apply
  the first inequality in the first chain to the
  first integral to get
  \[
    \int_{-\pi}^{\pi} f(x)\, dx
    = \int_{\pi}^{\pi + a} f(x)\, dx
    + \int_{-\pi + a}^{\pi} f(x)\, dx
    = \int_{-\pi + a}^{\pi + a} f(x)\, dx,
  \]
  as desired.
\end{proof}

\subsubsection{Exercise 2.2}
\begin{tcolorbox}
  In this exercise we show how the symmetries of
  a function imply certain properties of its Fourier
  coefficients. Let $f$ be a $2\pi$-periodic Riemann
  integrable function defined on $\R$.
  \begin{enumerate}
    \item Show that the Fourier series of the function
      $f$ can be written as
      \[
        f(\theta) \sim \hat{f}(0)
        + \sum_{n \ge 1} \left([\hat{f}(n) + \hat{f}(-n)] \cos n\theta + i [\hat{f}(n) - \hat{f}(-n)] \sin n\theta\right).
      \]
    \item Prove that if $f$ is even, then
      $\hat{f}(n) = \hat{f}(-n)$, and we get a cosine
      series.
    \item Prove that if $f$ is odd, then
      $\hat{f}(n) = -\hat{f}(-n)$, and we get a sine
      series.
    \item Suppose that $f(\theta + \pi) = f(\theta)$
      for all $\theta \in \R$. Show that
      $\hat{f}(n) = 0$ for all odd $n$.
    \item Show that $f$ is real-valued if and only if
      $\overline{\hat{f}(n)} = \hat{f}(-n)$ for all
      $n$.
  \end{enumerate}
\end{tcolorbox}

\begin{proof}
  (a) From the usual representation of the Fourier
  series of $f$, we have
  \[
    f(\theta) \sim \sum_{n = -\infty}^\infty \hat{f}(n) e^{in\theta}
    = \hat{f}(0) + \sum_{n = 1}^{\infty} \hat{f}(n) e^{in\theta} + \sum_{n = -\infty}^{1} \hat{f}(n) e^{in\theta}
    = \hat{f}(0) + \sum_{n = 1}^{\infty} \left(\hat{f}(n) e^{in\theta} + \hat{f}(-n) e^{-in\theta}\right).
  \]
  Writing $e^{in\theta} = \cos n\theta + i \sin n\theta$
  and $e^{-in\theta} = \cos n\theta - i \sin n\theta$,
  we obtain
  \[
    f(\theta) \sim \hat{f}(0)
    + \sum_{n = 1}^{\infty} \left([\hat{f}(n) + \hat{f}(-n)] \cos n\theta + i [\hat{f}(n) - \hat{f}(-n)] \sin n\theta\right)
  \]
  by grouping the real and imaginary parts.

  (b) If $f$ is even, then $f(\theta) = f(-\theta)$ and
  thus
  \[
    \hat{f}(-n) = \frac{1}{2\pi} \int_{-\pi}^{\pi} f(\theta) e^{in\theta}\, d\theta
    = \frac{1}{2\pi} \int_{-\pi}^{\pi} f(-\theta) e^{-in\theta}\, d\theta
    = \frac{1}{2\pi} \int_{-\pi}^{\pi} f(\theta) e^{-in\theta}\, d\theta
    = \hat{f}(n)
  \]
  after making the change of variables
  $\theta \mapsto -\theta$ (note that we pick up a minus
  sign from the Jacobian but the bounds get reversed,
  so the two effects are canceled out). As a result,
  we have $\hat{f}(n) - \hat{f}(-n) = 0$ and the
  sine terms in the series vanish, so we are
  left with a cosine series.

  (c) If $f$ is odd, then $f(\theta) = -f(-\theta)$ and
  everything is the same as in (b) except we pick up
  an extra minus sign in the third equality, so
  we end up with $\hat{f}(-n) = -\hat{f}(n)$. As a result,
  we have $\hat{f}(n) + \hat{f}(-n) = 0$ and the
  cosine terms vanish instead, so we are
  left with a sine series.

  (d) Let $n$ be odd and look at
  \[
    \hat{f}(n)
    = \frac{1}{2\pi} \int_{-\pi}^{\pi} f(\theta) e^{-in\theta}\, d\theta
    = \frac{1}{2\pi} \int_{-\pi}^{0} f(\theta) e^{-in\theta}\, d\theta
    + \frac{1}{2\pi} \int_{0}^{\pi} f(\theta) e^{-in\theta}\, d\theta.
  \]
  In the first integral, make the change of
  variables $\theta \mapsto \theta - \pi$ to get
  \[
    \frac{1}{2\pi} \int_{-\pi}^{0} f(\theta) e^{-in\theta}\, d\theta
    = \frac{1}{2\pi} \int_{0}^{\pi} f(\theta - \pi) e^{-in(\theta - \pi)}\, d\theta
    = \frac{e^{\pi i n}}{2\pi} \int_{0}^{\pi} f(\theta) e^{-in\theta}\, d\theta
    = -\frac{1}{2\pi} \int_{0}^{\pi} f(\theta) e^{-in\theta}\, d\theta
  \]
  since $f(\theta - \pi) = f(\theta)$ by periodicity
  and $e^{\pi i n} = (-1)^n = -1$ for odd $n$.
  Substituting this back in yields
  \[
    \hat{f}(n)
    = -\frac{1}{2\pi} \int_{0}^{\pi} f(\theta) e^{-in\theta}\, d\theta.
    + \frac{1}{2\pi} \int_{0}^{\pi} f(\theta) e^{-in\theta}\, d\theta = 0,
  \]
  as desired.

  (e) $(\Longrightarrow)$ If $f$ is real-valued, then
  (we integrate over
  $[-\pi, \pi] \subseteq \R$, so
  the complex conjugate passes through)
  \[
    \hat{f}(-n)
    = \frac{1}{2\pi} \int_{-\pi}^{\pi} f(\theta) e^{in\theta}\, d\theta
    = \frac{1}{2\pi} \int_{-\pi}^{\pi} \overline{f(\theta)} e^{in\theta}\, d\theta
    = \frac{1}{2\pi} \int_{-\pi}^{\pi} \overline{f(\theta) e^{-in\theta}}\, d\theta
    = \overline{\frac{1}{2\pi} \int_{-\pi}^{\pi} f(\theta) e^{-in\theta}\, d\theta}
    = \overline{\hat{f}(n)}
  \]
  since $f(\theta) = \overline{f(\theta)}$ if and only if
  $f(\theta) \in \R$.

  (e) ($\Longleftarrow$) Now assume that
  $\overline{\hat{f}(n)} = \hat{f}(-n)$ for all $n$.
  By the second equality of the
  previous calculation this implies that
  \[
    \int_{-\pi}^{\pi} f(\theta) e^{in\theta}\, d\theta
    = \int_{-\pi}^{\pi} \overline{f(\theta)} e^{in\theta}\, d\theta
    \implies \int_{-\pi}^{\pi} (f(\theta) - \overline{f(\theta)}) e^{in\theta}\, d\theta = 0
  \]
  for every $n$. Letting $g(\theta) = f(\theta) - \overline{f(\theta)}$,
  we can rewrite this as $\hat{g}(n) = 0$ for all $n$.
  From here we can conclude
  \[0 = g(\xi) = f(\xi) - \overline{f(\xi)}\]
  whenever $g$ (and thus $f$) is continuous at $\xi$.
  Since $f$ is Riemann integrable, it is continuous
  almost everywhere, so $f \equiv \overline{f}$
  almost everywhere
  and thus $f$ is real-valued almost everywhere.
\end{proof}

\subsubsection{Exercise 2.10}
\begin{tcolorbox}
  Suppose $f$ is a periodic function of period $2\pi$
  which belongs to the class $C^k$. Show that
  \[\hat{f}(n) = O(1 / |n|^k) \quad \text{as } |n| \to \infty.\]
  This notation means that there exists a constant
  $C$ such that $|\hat{f}(n)| \le C / |n|^k$. We could
  also write this as $|n|^k \hat{f}(n) = O(1)$, where
  $O(1)$ means bounded.

  [Hint: Integrate by parts.]
\end{tcolorbox}

\begin{proof}
  We proceed by induction on $k$. When $k = 0$, note that
  $f$ is continuous on a compact set and thus uniformly
  continuous, so $f$ is Riemann integrable. Then
  (noting that $|e^{in\theta}| = 1$)
  \[
    |\hat{f}(n)|
    = \left|\frac{1}{2\pi} \int_{-\pi}^{\pi} f(\theta) e^{-in\theta}\, d\theta \right|
    \le \frac{1}{2\pi} \int_{-\pi}^{\pi} |f(\theta)|\, d\theta
    \le \frac{1}{2\pi} 2\pi \max_{-\pi \le \theta \le \pi} |f(\theta)|
    = \max_{-\pi \le \theta \le \pi} |f(\theta)|,
  \]
  where we know that the maximum exists since $f$ is
  continuous on a compact set. Thus $\hat{f}$ is bounded,
  i.e. $\hat{f}(n) = O(1)$. Now assume for induction
  that the result
  holds for some fixed $k \in \N$, i.e.
  that $\hat{g}(n) = O(1 / |n|^k)$ for any $2\pi$-periodic
  function $g \in C^k$, and we show that
  it remains true for any $2\pi$-periodic function
  $f \in C^{k + 1}$. Note that $k + 1 \ge 1$ for
  $k \in \N$, so that
  $f$ is at least once continuously differentiable.
  Thus we can integrate by parts
  (differentiating $f$ and integrating $e^{-in\theta}$)
  to see that
  \[
    \hat{f}(n)
    = \frac{1}{2\pi} \int_{-\pi}^{\pi} f(\theta) e^{-in\theta}\, d\theta
    = \frac{1}{2\pi} \left[f(\theta) \frac{e^{-in\theta}}{-in}\right]_{-\pi}^{\pi}
    + \frac{1}{2\pi in} \int_{-\pi}^{\pi} f'(\theta) e^{-in\theta}\, d\theta
    = \frac{1}{in} \widehat{f'}(n),
  \]
  where the first term vanishes since
  $f$ and $e^{-in\theta}$ are both $2\pi$-periodic.
  Now since $f$ is $2\pi$-periodic and $(k + 1)$-times
  continuously differentiable, $f'$ is also
  $2\pi$-periodic but only $k$-times continuously
  differentiable. So we can apply the induction
  hypothesis to conclude that
  $\widehat{f'}(n) = O(1 / |n|^k)$. From here it follows
  that
  \[
    |\hat{f}(n)|
    = \left|\frac{1}{in} \widehat{f'}(n)\right|
    = \frac{1}{|n|} \left|\widehat{f'}(n)\right|
    \le \frac{1}{|n|} \frac{C}{|n|^{k}}
    = \frac{C}{|n|^{k + 1}},
  \]
  for some constant $C$ since
  $\widehat{f'}(n) = O(1 / |n|^k)$. Thus
  $\hat{f}(n) = O(1 / |n|^{k + 1})$,
  and we finish by induction.
\end{proof}

\end{document}
